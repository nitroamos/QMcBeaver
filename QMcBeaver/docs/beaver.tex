\documentclass[11pt]{article}
\usepackage{bm}
\usepackage{graphicx}
\usepackage{amsmath}
\usepackage{calc}
\usepackage{amssymb}
\usepackage{amsbsy}
\usepackage{verbatim}
\usepackage{algorithm}
\usepackage{algorithmic}

%\usepackage[labelsep=period, aboveskip=0pt, belowskip=0pt]{caption}
\usepackage{caption}

\oddsidemargin 0.0in
\topmargin -0.5in
\headheight 0.0in
\textwidth 6.5in
\textheight 9in

\begin{document}
\title{Getting to know the Beaver:\\ QMcBeaver for Novices and Experts
  Alike}
\author{Chip Kent, Mike Feldmann, Rick Muller, Amos Anderson, and Dan Fisher}
\maketitle

\newcommand{\NN}{\mbox{$\mathbb N$}}
\newcommand{\RR}{\mbox{$\mathbb R$}}

\section{Introduction: What is QMcBeaver?}
QMcBeaver is a finite, all electron variational and diffusion quantum
Monte Carlo program.  It was written by Chip Kent and Mike Feldmann at
Bill Goddard's Materials and Process Simulation Center at the
California Institute of Technology as part of their thesis work. The
program is currently being developed by Amos Anderson and Dan Fisher
at Caltech. The code is currently
being distributed under the Gnu General Public License.

Several features distinguish QMcBeaver from other QMC programs:
\begin{itemize}
\item As already mentioned, the code is distributed free of charge
  under the GPL, which allows anyone to use or to modify the code, and
  also makes the algorithms that QMcBeaver uses open for review by
  anyone.
\item The code runs on massively parallel architectures, and has a
  highly efficient decorrelation algorithm to properly decorrelate the
  data from different processors~\cite{DDDA}.
\item The code uses a Manager-Worker based parallelization scheme
  which allows the efficient use of heterogeneous clusters~\cite{QMC-MW}.
\item The code is written in modern object-oriented C++, and is clean,
  and easily understood and modified.

% Guys: do you want to list anything more here? The MC stuff?

\end{itemize}

This document will serve as a guide to getting started using
QMcBeaver. We assume that the user has some familiarity with basic
quantum chemistry methods such as Hatree-Fock (HF) or Density
Functional Theory (DFT). Additionally, some familiarity with the
techniques of variational and diffusion QMC will be required to
appreciate this guide.

\section{Introduction to Quantum Monte Carlo}
There are a couple good introductions to the subject material if you are unfamiliar.
\emph{Modern Quantum Chemistry} by Szabo and Ostlund\cite{Szabo82} is a good place to start
reading about Hartree-Fock and the relevant linear algebra. As a bonus, this book is published by Dover, so it's cheap.
The book \emph{Computational Physics} by Thijssen\cite{Thijssen99} has a good chapter on QMC. If you want to
play with some toy code, covering both of these methods, look for the Sourceforge project PyQuante. Chip Kent's thesis
is also a good reference\cite{kent2003}, including a discussion of the algorithms he and Mike developed.
In addition, pasted here is an introduction to the code from a computational aspect.

\newcommand\colr{\bar{\bm r}}
The most important information about a molecule is its ground state
energy, calculated by means of the time independent Schr{\"o}dinger
equation
\begin{equation} \label{eq:schro}
\left< E \right> = \frac{ \int \Psi(\colr) \hat{H} \Psi(\colr) d\colr }
{  \int \Psi^2(\colr) d\colr }
\end{equation}
where $\Psi(\colr):\mathbb{R}^{3N} \rightarrow \mathbb{R}$ is the wave
function, mapping the $3N$ Cartesian coordinates of $N$ electrons into
a probability amplitude related to the probability density in
Eq.~\eqref{eq:distribution}.  (Equation~\eqref{eq:schro} includes the
common restriction that $\Psi(\colr)$ is a real valued function.) The
Hamiltonian operator $\hat{H}$ is given by
\begin{equation} \label{eq:hamiltonian}
\hat{H} = \nabla^2 + V(\colr)
\end{equation}
where the Laplacian is over all $3N$ electronic coordinates and
calculates the kinetic energy (in the unitless Hartree measure) of the
electrons in the molecule.  The $V(\colr)$ term represents the
potential energy due to Coulomb interactions between all pairs of
electrons and nuclei. The energy $E$ is the eigen value of $\hat{H}$
operating on the eigen function $\Psi(\colr)$. The ground state energy
is the lowest such eigen value, and is of primary interest here.

There are many methods to calculate
Eq.~\eqref{eq:schro} with varying degrees of accuracy and
computational complexity. The highly accurate QMC family of
algorithms~\cite{ReynoldsCeperleyAlder1982} uses
Metropolis~\cite{Metropolis} integration to fine tune the result
provided by a cheaper method. It uses the \emph{local energy}
\begin{equation} \label{eq:elocal}
E_L (\colr) = \frac{\hat{H} \Psi(\colr)}{\Psi(\colr)} =
\frac{\nabla^2 \Psi(\colr)}{\Psi(\colr)}
+ V(\colr)
\end{equation}
which represents an evaluation of the energy for a set of electronic
coordinates. In terms of the stationary probability distribution of
electrons
\begin{equation} \label{eq:distribution}
\rho(\colr) = \frac{\Psi^2(\colr)}{  \int \Psi^2(\colr) d\colr }
\end{equation}
we can transform Eq.~\eqref{eq:schro} into the Monte Carlo integration
form
\begin{equation} \label{eq:montecarlo}
\left< E \right> = \int \rho(\colr) E_L(\colr) d\colr
= \lim_{N_t \to \infty} \frac{1}{N_t} \sum_{t=1}^{N_t} E_L(\colr_t).
\end{equation}
Here $\colr_t$ are a series of electronic coordinates generated with
respect to $\rho(\colr)$ by some importance sampling
scheme~\cite{UmrigarNightingaleRunge1993}.
%PS: not important for us
%% , which incorporates the
%% ability to asymptotically correct most deficiencies in the input
%% wave function.
Since error scales as $1/\sqrt{N_t}$ in Monte Carlo methods a rather
large number of samples is required to achieve useful accuracies.
Additionally, it is common to run several independent series, called
\emph{walkers}, in order to minimize the error due to serial correlation
between the $N_t$ data points.

In terms of computational complexity, the difficulty for QMC lies in
the evaluation of $\nabla^2 \Psi(\colr_t)$ for each $E_L(\colr_t)$ as
well as the evaluation of $\Psi(\colr_t)$ and $\nabla \Psi(\colr_t)$
which are used for importance sampling.  The most common functional
form for $\Psi(\colr)$ has at least three nested stages of evaluation.
At the first stage, we place a collection of $N_{bf}$ basis functions
centered at the nuclei in the $3D$ coordinate space. Typically a given
nucleus is associated with multiple basis functions. The basis function
takes as argument the local coordinates of a given electron ($i$) relative
to the nucleus ($j$), $\vec{r}_{ij} =\vec{r}_i - \vec{R}_j$. The best
results are achieved with the following functional form
\begin{equation} \label{eq:basisf}
\chi_{j} (x_{ij},y_{ij},z_{ij}) = x^{k_j}_{ij} y^{l_j}_{ij} z^{m_j}_{ij}
\sum_{n_j} a_{n_j} e^{-b_{n_j} r^2_{ij}}.
\end{equation}
For each basisfunction, $R_j$, $k_j$, $l_j$, $m_j$, $n_j$, $a_{n_j}$
and $b_{n_j}$ are parameters given as input to the QMC program. The
$k_j, l_j, m_j \in \NN$ parameters give the basisfunction the required
symmetry, and $n_j \in \NN^{+}$ helps select the quality of fit. The
other parameters are all real numbers.

The second stage of evaluation takes linear combinations of
basisfunctions to create \emph{molecular orbitals}. The
$k^\textrm{th}$ orbital is given by $\phi_{k}( \vec{r}_i ) = \sum_j
\chi_{j} (r_{ij}) c_{jk}$, where $c_{jk} \in \RR$ are coefficients
input to QMC. These orbitals represent the spread of the electron
across the entire molecule.

Finally, the third stage of evaluation relevant to this study is the
\emph{Slater determinant}, chosen for its antisymmetric properties.
For the $N_s$ electrons of a given quantum spin $\alpha$ or $\beta$
($N = N_\alpha + N_\beta \sim 2 N_\alpha$) the determinant is a
function of the $\phi_k$ (which in turn are functions of the $\chi_{j} (r_{ij}) $)
\begin{equation} \label{eq:slaterd}
D_{s} (\colr_s)  = |M_s(\colr_s)| = \begin {vmatrix}
\phi_1 \left( \vec{r}_1 \right) & \phi_2 \left( \vec{r}_1 \right)
& \cdots & \phi_N \left( \vec{r}_1 \right) \\
\phi_1 \left( \vec{r}_2 \right) & \phi_2 \left( \vec{r}_2 \right) & & \\
\vdots & & \ddots & \\
\phi_1 \left( \vec{r}_{N_s} \right) & & & \phi_{N_s} \left( \vec{r}_{N_s} \right)
\end{vmatrix}
\end{equation}
(here we partition $\colr$ into $\colr_\alpha$ and $\colr_\beta$)
and the wavefunction is
\[
\Psi(\colr) = D_\alpha(\colr_\alpha)D_\beta(\colr_\beta).
\]
To calculate the kinetic energy, we first obtain $\nabla^2_i \phi_{k}(
\vec{r}_i ) = \sum_j \nabla^2_i \chi_{j} (\vec{r}_{ij}) c_{jk}$, and
then sum the contributions from all the electrons in all the orbitals
%PS: this looks fine; it's a big matrix multiply... (just imagine
%PS: stuffing the Laplace terms appropriately into a matrix and you
%PS: would have an exact matrix multiply. But writing it out is easier
%PS: than to set things up correctly in terms of the Laplace matrix
%PS: (from what I can tell) because of the \alpha/\beta business
\begin{equation} \label{eq:matrixdot}
\frac{\nabla^2 \Psi(\colr)}{\Psi(\colr)} =
\sum_{s \in \{\alpha, \beta\}} \sum_{i,k \in N_s}
%The point of introducing "M_s" before was to have a label to reference here
\left[M_s^{-1}(\colr_s)\right]_{ki}
%\left\{\phi_{k'} \left( \vec{r}_{i'} \right)\right\}^{-1}_{ki}
%\left\{\phi_{k'} \left( \vec{r}_{i'} \right)\right\}^{-1}_{ki}
\nabla^2_i \phi_{k}( \vec{r}_i ).
\end{equation}
A similar procedure is followed for calculating the gradient of the
wavefunction for each electron with the exception that the final
summation results in a vector of gradients.

To summarize the algorithm, we are given a set of
nuclear coordinates, basis function parameters (available in the
literature), and the $c_{jk}$, which describe the wavefunction as fit
by some other (more approximate and cheaper) method. Additionally, we
choose some parameters including the number of steps $N_t$, the number
of walkers $W$, an initial guess scheme for positions $\colr$ of all
the electrons, as well as several parameters relating to the
importance sampling. With these in hand, the algorithm can be stated as
shown in Algorithm 1 (the $\otimes$ represents matrix multiplication).
\begin{algorithm}
%I am having problems referencing this label... the \ref{qmcalg} was not working
%PS: yeah. I've had problems with labels in algorithm. I don't
%PS: remember anymore whether I found a fix of just gave up though...
\caption{The QMC algorithm} \label{qmcalg}
\begin{algorithmic}
\STATE $E_{sum} \leftarrow 0$
\FOR{$w=1$ to $W$}
\STATE $\vec{r}_{ij} \leftarrow $ initialize()
\FOR{$t=1$ to $N_t$}
\FOR{$s=\alpha$ and $s=\beta$}
\STATE $M_s \leftarrow \chi_{j} (\vec{r}_{ij}) \otimes c_{jk}$
\STATE $X_s \leftarrow \frac{\partial}{\partial x_{i}} \chi_{j} (\vec{r}_{ij}) \otimes c_{jk}$
\STATE $Y_s \leftarrow \frac{\partial}{\partial y_{i}} \chi_{j} (\vec{r}_{ij}) \otimes c_{jk}$
\STATE $Z_s \leftarrow \frac{\partial}{\partial z_{i}} \chi_{j} (\vec{r}_{ij}) \otimes c_{jk}$
\STATE $L_s \leftarrow \nabla^2_i \chi_{j} (\vec{r}_{ij}) \otimes c_{jk}$
\ENDFOR
\STATE Jastrow $\leftarrow J(\colr)$
\STATE $\Psi \leftarrow \det M_\alpha * \det M_\beta * $ Jastrow
%\STATE $E_{sum} \leftarrow E_{sum} +
%M^{-1}_\alpha \odot L_\alpha +
%M^{-1}_\beta \odot L_\beta +
%V(\colr) $
%\STATE //\textit{All derivatives are needed due to the chain rule}
\STATE $E_{sum} \leftarrow E_{sum} + E_L(M_s, \text{Jastrow},\{\textit{derivatives}\}...)$
%\STATE $\vec{r}_{ij} \leftarrow $ sampling($\Psi, \vec{r}_{ij}, 
%X_{\alpha,\beta}, Y_{\alpha,\beta}, Z_{\alpha,\beta}, L_{\alpha,\beta}$)
\STATE $\vec{r}_{ij} \leftarrow $ sampling($\Psi, \vec{r}_{ij}, X_s, Y_s, Z_s, L_s$)
\ENDFOR
\ENDFOR
%for some reason, my latex compiler is not accepting a \RETURN here
\STATE $E_{avg} \leftarrow E_{sum}/(N_t * W)$
\end{algorithmic}
\end{algorithm}
The high degree of parallelism is evident since each processor can
calculate all the linear algebra for its walkers and only needs to
produce a single value; the energy. While some QMC algorithms only
update one electron per Monte Carlo step, our method updates
all at once~\cite{UmrigarNightingaleRunge1993}.

One big advantage of QMC relative to alternative methods is the
freedom one has in choosing the functional form of $\Psi(\colr)$. We
exploit this by multiplying the Slater determinant wave function with
a set of pairwise interaction terms which explicitly model some of the
underlying physics. The only condition is that these terms, called
\emph{Jastrow} functions, preserve the antisymmetry of the wave
function. To satisfy this condition, we use the functional form
\begin{equation} \label{eq:jastrow}
J(\colr) =  \prod_{q<p} e^{ u_{pq} (r_{pq})}
\end{equation}
which provides a term for each particle-particle interaction, where 
\begin{equation} \label{eq:jastrowU}
u_{pq} (r_{pq}) =
\frac{ \sum_{\kappa=1}^{\Gamma} a_{pq\kappa} r_{pq}^\kappa}
{1+\sum_{\kappa=1}^\Lambda b_{pq\kappa} r_{pq}^\kappa}
\end{equation}
and $p$ and $q$ index all electrons and nuclei, and $r_{pq}$ is the
distance separating the two particles. The number of terms ($\Gamma$
and $\Lambda$) is arbitrary, and depends on the quality of fit.  These
parameters, along with $a_{pq\kappa}, b_{pq\kappa} \in \RR$, are input
to the QMC algorithm. With this modification, our wave function is now
$\Psi_{QMC}(\colr) = D_\alpha(\colr_\alpha) D_\beta(\colr_\beta)
J(\colr)$, and there are chain rule effects for the gradient and
Laplacian. The rationale for these additional terms is the improved
convergence if the wave function is a better approximation of the
eigen function of $\hat{H}$ to begin with.

\section{Compiling QMcBeaver}
If for some strange reason you choose to compile this using a program like Microsoft Visual Studio, then this section is irrelevant. The included Visual Studio project file in the {\texttt windows} directory should set up the program appropriately, although some effort may be required to change some of the options for debugging or ATLAS. Visual Studio by default uses the \texttt{cl.exe} compiler produced by Microsoft itself. We have not tried the \texttt{icc} compiler from Intel. A parallel version of QMcBeaver for Windows has not been explored as there are no free MPI (message passing interface) packages available to try this out on. Finally, with respect to the Windows platform, installing cygwin is a good option to try. Cygwin is free -- it installs a ''linux aware'' environment running within Windows.

There are 3 important files in the process of compiling QMcbeaver; two are given, one is created. The python script
\texttt{configure.py} will create the \texttt{Makefile.config} file, which is then used in conjunction with the \texttt{Makefile} script to compile the executable. There are a couple macros used in the code, and these will be discussed.

\subsection{configure.py}
There are a couple important options with this script. If you are in the QMcBeaver directory, typing {\texttt./configure.py} will list the options to your screen.
\begin{enumerate}
\item You must specify a compiler to use. Most machines have \texttt{gcc} installed, but some have vendor provided compilers. Our \texttt{configure.py} knows about many of them, if you are attempting to compile on a machine that we have not tried, then by comparing with our other compiler configurations, you should be able to guess a good set.

\item QMcBeaver requires parallel processing to fully take advantage of its algorithm. The parallel portions of QMcBeaver has been programmed using the MPI. This code is usually run and tested on LAM/MPI, but it should work on MPICH as well without any problem. You must select \texttt{MPICC} as your compiler. In addition, there might be some steps necessary to ensure that \texttt{mpicc} and relevant libraries are on your path.

\item There are a couple variations that can change QMcBeaver's behavior. None of them are required.
\begin{itemize}
\item {\texttt --atlas}: The ATLAS\cite{WN147, atlas_siam} library is a freely available Basic Linear Algebra Subroutines (BLAS) package available at Sourceforge. The advantage ATLAS provides over vender BLAS libraries is that it automatically tunes itself to each processor during compilation, and is thus cross-platform compatible. Essentially, this library can be used to handle the matrix-matrix multiplication used in QMcBeaver. This makes a substantial difference for larger molecules (larger basis sets or many orbitals). For smaller molecules and atoms, ATLAS will not provide much of an advantage and could possibly slow the execution time since the cost of calling the library would come into effect.  We have not successfully compiled this library everywhere. On the other hand, there are precompiled libraries for many systems available online. While ATLAS should work everywhere, sometimes it will conflict with a debugger. Instructions for installing ATLAS are included in the {\texttt lib} directory. Using ATLAS with Visual Studio requires a few special instructions, like requiring changing the library names from \texttt{libatlas.a} to \texttt{atlas.lib} etc. Google has more information.
\item {\texttt --debug}: This option is probably only useful for developers. Including this option will place debug symbols in your code, useful when QMcBeaver is run with a program like {\texttt gdb} or whatever has been installed on your system. Note, this option sometimes conflicts with optimization flags. Specifying both may produce unpredictable results. For example, the compiler may choose between the optimize and debug options.
\item {\texttt --optimize}: This option should select the best set of compiler settings for the given architecture and compiler type. However, as this is largely a black art, hand-tweaking the options in {\texttt configure.py} may improve the executable. Since it is turned on by default for most compilers, you do not need to include this flag.
\item {\texttt --profile}: This option is probably only useful for developers. It will produce an extra output file when the executable is run. This output is useful with a program like {\texttt gprof} for analyzing QMcBeaver's calling tree for possible optimizations. 
\item {\texttt --float}: The key parts of the code can be switched to running with single precision variables as opposed to the typical double precision variables. Single precision will hurt the accuracy of the result, but the code will run faster on many systems.
\item {\texttt --vtune}: To profile the code using Intel vTune, some special parameters must be used to obtain "position independent code".
\item {\texttt --gpu}: This is used to turn on the graphics card portions of the code. It is similar to using ATLAS. 
\item {\it tag}: You can tag your executables with the tag \textit{tag}. The purpose is that if you are playing with different options, you can recompile all code for a given \textit{tag} with the command \texttt{make updateall \textit{tag}}.
\end{itemize}
\end{enumerate}

\subsection{Makefile.config}
This file is generated by {\texttt configure.py}, and for most non-developers, this file can be safely ignored. This is the easiest place to start when toying with different compiler options.

\subsection{Makefile}
Once {\texttt configure.py} has been run and the {\texttt Makefile.config} file has been generated, simply enter the command {\texttt make} on the command line in the QMcBeaver directory. The {\texttt make} program will use the rules in the {\texttt Makefile} with the options in {\texttt Makefile.config} to create the executable, which will be placed in the {\texttt bin} directory. However, there are a few options given to {\texttt make} by the {\texttt Makefile} we provide that should be listed. When making QMcBeaver, {\texttt make} will create a separate directory for intermediate files for each type of {\texttt Makefile.config} created. For example, if the ATLAS option is specified through {\texttt configure.py}, a directory appropriately named will be created. If later {\texttt configure.py} is run without the ATLAS option, or with a different compiler, or with the debug option, or any difference in options, a new directory will be created for the temporary files. This is useful when the same home directory is used for multiple computers with different operating systems, like in the Goddard group. This means the compiler does not have to recreate all the temp files every time an option is changed, if that option has been compiled before.

\begin{itemize}
\item {\texttt bruteforce}: This option might help with some stubborn compilers. It ignores most of the options in {\texttt Makefile.config}
\item {\texttt --clean}: This option will remove the executable and all intermediate compiler files associated with {\texttt Makefile.config}. That is, it will only clean one option set.
\item {\texttt --cleanall}: This will remove the intermediate files for all the compiler options.
\item {\texttt --updateall \textit{tag}}: In the process of modifying code, the developer may wish to test a set of different compiler options all at once. Instead of forcing that developer to systematically recreate each {\texttt Makefile.config} and recompile, this option will recompile all the options that have been previously compiled with tag \textit{tag}. This is possible because all of the compiler options were stored in the intermediate files when that option set was originally compiled.
\end{itemize}


\section{Simple Atomic Calculations: Helium}
% The idea here is to introduce the program in the simplest way, and
% with only the most important functions. More complicated examples,
% to follow, will show more of the breadth of the available options.
We will start with a simple example: Helium atom using a 6-31G basis
set. Helium is one of the few system whose exact answer is known
(-2.9037 h), and so provides a good test of how well QMC is working.
\subsection{Generating a Trial Wave Function}
\subsubsection{Using GAMESS to Generate a Trial Wave Function}
Our first step is to run a GAMESS calculation for He. A sample GAMESS
input deck looks like
\begin{verbatim}
 $CONTRL SCFTYP=RHF RUNTYP=ENERGY COORD=CART UNITS=ANGS $END
 $BASIS  GBASIS=N31 NGAUSS=6 $END
 $GUESS  GUESS=HUCKEL $END
 $DATA
He ground state
Dnh 2

He   2.0       0.0000     0.0000     0.0000 
 $END
\end{verbatim}
The GAMESS RHF calculation converges with an energy of -2.8552 h.

We can then run the script 
\begin{verbatim}
% (QMcBeaver)/bin/gamess2qmcbeaver.py he.out
\end{verbatim}
The variable \verb-QMcBeaver- should be set to the root 
directory of the QMcBeaver program.  This script will convert the GAMESS
output to a \verb-he.ckmf- input file for QMcBeaver. This file is
described below. 

\subsubsection{Using Jaguar to Generate a Trial Wave Function}
A Jaguar input file to do the same He calculation looks like
\begin{verbatim}
&gen
basis=6-31G
ip164=2
&
&zmat
He1     0.0000000000000   0.0000000000000   0.0000000000000
&
\end{verbatim}

The flag \verb-ip164=2- tells Jaguar to write a he.bas file with the
basis set for the calculation, which is needed to create the QMcBeaver
input.  We can then run the script
\begin{verbatim}
% (QMcBeaver)/bin/jaguar2qmcbeaver.py he.01.in
\end{verbatim}
For Jaguar calculations, the argument is the restart file, not the
output file.

\subsection{Structure of the simple he.ckmf input file}
The \verb-he.ckmf- file produced by the QMcBeaver conversion scripts
has the form
\begin{verbatim}
&flags
atoms
 1
charge
 0
energy
 -2.8551604262
norbitals
 2
nbasisfunc
 2
run_type
 variational
chip_and_mike_are_cool
 Yea_Baby!
&
&geometry
HE      2       0.000000        0.000000        0.000000
&
&basis
HE      2       3
        3       s
                38.4216340      0.040139739346
                5.7780300       0.261246097041
                1.2417740       0.79318462463
        1       s
                0.2979640       1.0
&
&wavefunction
1 1
        0.592081                0.513586        

0 0
        -1.149818               1.186959        

Alpha Occupation
1       0       

Beta Occupation
1       0       

CI Coeffs
1.0

&

&Jastrow

ParticleTypes: Electron_Up Electron_Down
CorrelationFunctionType: FixedCuspPade
NumberOfParameterTypes: 2
NumberOfParametersOfEachType: 0 1
Parameters: 3.0
NumberOfConstantTypes: 1
NumberOfConstantsOfEachType: 1
NumberOfConstantTypes: 1
NumberOfConstantsOfEachType: 1
Constants: 0.5

ParticleTypes: Electron_Up HE
CorrelationFunctionType: FixedCuspPade
NumberOfParameterTypes: 2
NumberOfParametersOfEachType: 0 1
Parameters: 100.0
NumberOfConstantTypes: 1
NumberOfConstantsOfEachType: 1
Constants: -2.0

ParticleTypes: Electron_Down HE
CorrelationFunctionType: FixedCuspPade
NumberOfParameterTypes: 2
NumberOfParametersOfEachType: 0 1
Parameters: 100.0
NumberOfConstantTypes: 1
NumberOfConstantsOfEachType: 1
Constants: -2.0

&Jastrow
\end{verbatim}
This file is mostly self-explanatory, but we'll cover a few points
that are worth emphasizing nonetheless.

The \emph{flags} section contains general input parameters, including
the number of atoms, the molecular charge, the number of orbitals and
basis functions. Note that it also contains the \emph{run\_type} flag,
currently set to \emph{variational}, indicating that we will first be
doing VQMC calculations.  The \emph{flags} section contains all of the
parameters that the user specifies to control the calculation.  The
flags what they do are described in a later section.

The \emph{geometry} section contains the geometry of the molecule.
Cartesian coordinates in atomic units (Bohr) are given for each nucleus.
The \emph{basis} section defines the two basis functions we will be
using, and the \emph{wavefunction} section describes the coefficients
of the trial molecular orbitals in terms of these basis functions.

The \emph{Jastrow} section is worth describing in slightly more
detail. This section details the correlation functions we will use in
the QMC calculations to follow.  A correlation function is defined for
each pair of particles in the molecule.  In the case of the He atom,
we have electron-up-electron-down, electron-up-He, and
electron-down-He correlation functions.  If there were more than one
up electron, there would be an electron-up-electron-up correlation
function, and so on.   

In this example, all functions are set to the \emph{FixedCuspPade}
form, and guesses as to the appropriate parameters are included. 
The format for the constants and paramters in this and other available
correlation function forms will be demonstrated in later sections of
this manual.

\subsection{Running a VQMC Calculation}
A variational QMC calculation allows us to statistically sample the
input wave function, including the Jastrow terms. By running several
of these calculations, we can optimize the parameters in the Jastrow
function and get a more accurate approximation to the true energy.

We can run a simple VQMC calculation via the command
\begin{verbatim}
% $(QMcBeaver)/bin/QMcBeaver.linux he.ckmf
\end{verbatim}

This command will run 1,000,000 steps of VQMC using the Jastrow
parameters specified in the input ckmf file, and will then write
several files, including a log of various quantities every 1000 steps
of the simulation (he.qmc), a summary of the statistics for the run
(he.rslts), and a new input file (he.01.ckmf) and a checkpoint file
that can be used to restart the calculation.

This example calculation is only an evaluation of the trial
wavefunction defined in the .ckmf file.  The optimization parts of the
code are turned off, so all of the parameters are fixed at their
default values.  

Several points are worth investigating in the he.rslts. We first
consider the energy, which is reported to be $-2.6337228773 \pm
0.1738595133$ h. This is significantly worse than either the HF
(-2.855) or exact (-2.903) energies for several reasons. First, the
statistics aren't converged, as indicated by the variance estimate of
$\pm 0.1738595133$. The size of this number indicates that not even
the first decimal place of the total energy is converged, and that we
will have to run a significantly longer calculation to obtain higher
precision. 

If we simply want higher precision, we can restart the VQMC
calculation using the command
\begin{verbatim}
% $(QMcBeaver)/bin/QMcBeaver.linux he.01.ckmf
\end{verbatim}
% This gave garbage: Is it not the thing to do?

Another statistic that is reported in the he.rslts file is the Virial
Ratio. From the Virial Theorem we know that in a system of particles
interracting through a Coulomb potential, the ratio of the average
potential energy to the average kinetic energy,
$-\langle V\rangle/\langle T \rangle$, is 2. The fact that the program
reports this ratio as $2.6311988719 \pm 0.3675653079$ is another sign
that the statistics are not sufficiently converged.

But the most important cause of error in this result is the fact that
we have not optimized the Jastrow parameters that go into the
calculation, we simply used the guess values that the conversion
program gave us.  Optimizing the parameters in the wavefunction is
covered in later sections of the manual.

\subsection{Running a DQMC Calculation}
Once we have sufficiently converged the Jastrow parameters via VQMC,
we are ready to begin a DQMC calculation that can further optimize the
energy by allowing the walkers to breed/die or reweight themselves
based upon the true potential. A good rule of thumb is that the
Jastrow parameters have been sufficiently converged when the variance
is 50\% of the variance from the HF input guess.
% Is this estimate still roughly correct?

% Put in an aside here about why Jastrow optimization is important 
% before doing DQMC calculations

% Edit the input file, changing 'variational' to 'diffusion'
% Choose the type of branching
% Run the calculation.
% How can we tell when we're getting garbage?

\section{A Slightly More Complicated Example: Water}
To be completed.
% The idea here is to gradually introduce more ideas and input options.

% Any other examples to include?? I think we need to demonstrate all
% of the features we want people to know about. Things that we might
% want to detail include:
%
% Different initialization schemes
% Three-body Jastrow functions
% Multiconfigurational reference functions
%  The MC functions might be a good way to do a hard calculation 
%   like N2 dissociation.

\section{The Flags}
This section lists the flags that can be specified in the .ckmf file.

\subsection{The Molecule}

\begin{description}
\item [Natoms] The number of atoms in the molecule.
\item [charge] Charge of the system in atomic units ($+1$ if a neutral
  molecule loses an electron).
\item [energy] The initial trial energy in hartree au for the molecule.
\end{description}

\subsection{The Trial Wavefunction}

\begin{description}

\item [Norbitals] Total number of orbitals (occupied and unoccupied) in
  the SCF part of the trial wavefunction.

\item [Nbasisfunc] Number of basis functions used to create the
  orbitals of the trial wavefunction. 

\item [Ndeterminants] Number of Slater determinants in the SCF part of
  the wavefunction.

\end{description}

\subsection{Sampling}

\begin{description}

\item [walker\_initialization\_method] Specifies the method by which the
  initial electronic configurations for the walkers are generated.
  The default option is \verb-mikes_jacked_walker_initialization-, which
  assigns the electrons for each atom randomly to a spherical gaussian
  centered on the atom.  The other option is
  \verb-dans_walker_initialization-, which uses STRAW~\cite{STRAW} to
  generate initial configurations that need very few equilibration
  steps.

\item [iseed] The initial seed for the random number generator.
  The random number generator used in the program is ran1 from
  Numerical Recipes~\cite{Press93}.  This number must be a
  negative integer.

\item [sampling\_method] This is the method by which new configurations
  are proposed and either accepted or rejected.  The first option is
  \verb-no_importance_sampling-, which means that the displacement of
  each electron in each direction is distributed with respect to a
  Gaussian whose standard deviation is the square root of the time
  step.  The second option is \verb-importance_sampling-, in which the
  electrons are first displaced according to the quantum force and
  then diffuse randomly.  The third option is
  \verb-umrigar93_importance_sampling-, which uses Umrigar's
  accelerated Metropolis method~\cite{UmrigarNightingaleRunge1993}.  

\item [QF\_modification\_type] \verb-none- if the quantum force,
  $\frac{\nabla \Psi}{\Psi}$, is not to be modified.  Other
  possibilities are \verb-umrigar93_equalelectrons- and
  \verb-umrigar93_unequalelectrons-, both of which are described
  in~\cite{UmrigarNightingaleRunge1993}.  

\item [energy\_modification\_type] \verb-none- if the local energy
  is not to be modified when weighting the walkers.  Other
  possiblities are \verb-modified_umrigar93- and \verb-umrigar93-,
  both of which are described in~\cite{UmrigarNightingaleRunge1993}.

\item [umrigar93\_equalelectrons\_parameter] If Umrigar's 1993 QMC
  algorithm~\cite{UmrigarNightingaleRunge1993} is used with the equal-electron
  modifications to the local energy and quantum force, this is the
  parameter describing how the modifications will occur.  This
  parameter can be in the range $(0,1]$, and the default value is 0.5.

\item [walker\_reweighting\_method] Algorithm used to reweight the
  walkers after a time step.  The possibilities are
  \verb-simple_symmetric-, \verb-simple_asymmetric-, and
  \verb-umrigar93_probability_weighted-, all of which are described
  in~\cite{UmrigarNightingaleRunge1993}. 

\item [branching\_method] Method for branching walkers in DMC
  calculations.  There is no branching in VMC.  Possible
  methods are \verb-non_branching-, in which the weights are varied
  instead of branching, \verb-unit_weight_branching-, in which the
  walkers have integer weights~\cite{H+L}, and
  \verb-nonunit_weight_branching-, which is described
  in~\cite{UmrigarNightingaleRunge1993}.   

\item [branching\_threshold] If a walker's weight exceeds this
  threshold, it will branch when the branching method allows the
  walkers to have fractional weights.  The default value is 2.0.

\item [fusion\_threshold] If a walker's weight is less than this
  threshold, it will fuse with another low weight walker when the
  branching method allows the walkers to have fractional weights.  The
  default value is 0.5.

\item [old\_walker\_acceptance\_parameter] Parameter which can be
  used to prevent persistent configurations from interfering with a
  calculation~\cite{UmrigarNightingaleRunge1993}.

\item [dt] This is the time step used for the propagation phase
  of the calculation.  The default value is .001. Notes: There is sort of a balance here. If dt                                                                                                                       
     is large, then your error will be large. If dt is small, it will take longer to explore all the                                                                                                                    
     space. Officially, you're supposed to estimate an extrapolation                                                                                                            
     to dt = 0. That is, run it for (for example) 1e-3,                                                                                                                  
     then 1e-4, then 5e-5, etc until you can guess what                                                                                                                  
     the value at zero would be.  

\item [population\_control\_parameter] Parameter used in
  controlling the number of walkers or total weight of all walkers
  during a branching QMC calculation.  The default value is 1.

\item [correct\_population\_size\_bias] 1 if the correction for the
  population size bias for branching calculations is used, and 0
  otherwise. See~\cite{UmrigarNightingaleRunge1993}.

\end{description}

\subsection{Equilibration}

\begin{description}

\item [equilibration\_steps]  The number of equilibration steps that are
  to be taken before statistics are gathered.  This is necessary to
  allow the walkers to reach regions of high probability density, so
  that their configurations reflect the desired distribution.  

\item [dt\_equilibration]  The time step used during the equilibration
  phase of the calculation.  The default value is 0.02.
  Notes: You might want to set this higher than you would set your                                                                                                    
     regular dt since the point is to get the electrons in approximately                                                                                                 
     the right place, not to make measurements. 

\item [equilibration\_function]  The function used during the
  equilibration phase of the calculation.  The first option is 
  \verb-step-, in which the time step changes abruptly from
  \emph{dt\_equilibration} during the equilibration phase to \emph{dt}
  when equilibration is finished.  The second option is \verb-ramp-,
  in which the time step changes linearly from \emph{dt\_equilibration}
  to \emph{dt} during the equilibration phase.  The third option is
  \verb-CKAnnealingEquilibration1-, which is the same as \verb-step-
  except that for a certain number of steps, essentially every step
  proposed is accepted.  If importance sampling is used, this allows
  the walkers to reach high density regions very quickly.

\item [CKAnnealingEquilibration1\_parameter]  If
  \verb-CKAnnealingEquilibration1- is used, this is the number of
  steps taken in which every proposed displacement is accepted.  This
  number must be less than the number of equilibration steps.  
 
\item [equilibrate\_first\_opt\_step] 1 if the first VMC calculation used
  to generate correlated sampling configurations for optimization
  equilibrates for the specified equilibration time, and 0 otherwise.
  If a DMC calculation is performed, 1 if the calculation
  equilibrates, and 0 otherwise.   

\item [equilibrate\_every\_opt\_step] 1 if every VMC calculation used to
  generate correlated sampling configurations for optimization
  equilibrates, and 0 otherwise.  

\item [use\_equilibration\_array] 1 if the equilibration array is to be
  used and 0 otherwise.  This defines an array of statistics gathering
  objects, in which the $ith$ element becomes active on the $2^{i}th$
  iteration.  The object with the smallest standard deviation for the
  energy is chosen, which automatically determines the correct number
  of equilibration steps.  

\end{description}

\subsection{Controlling the Calculation}

\begin{description}

\item [run\_type] \verb-variational- or \verb-diffusion- QMC.

\item [temp\_dir] The temporary directory where correlated
  sampling configurations are to be written.

\item [max\_time\_steps] The maximum number of propagation steps to be
  carried out in the calculation.  In a parallel calculation, this is
  the total number of propagation steps on \emph{all} processors. 

\item [desired\_convergence] If the standard deviation of the energy in
     the global results falls below this level, sampling stops.
     Notes: It's difficult to determine a priori how many time                                                                                                           
     steps it will take to reach a given convergence, so it's often                                                                                                      
     easier to control the length of the run by setting this to 0.0                                                                                                      
     and choosing an appropriate max\_time\_steps.  

\item [number\_of\_walkers] The number of walkers on each processor.  At
  each time step, the statistics for the walkers on the processor are
  preblocked.  That is, for each observable, the average over the
  walkers on the processor becomes one sample.   

\item [output\_interval] At this interval on the root node, a summary of
  the global statistics are printed out.  Quantities reported include
  the global energy, its standard deviation, and the total number of
  samples collected. 

\item [mpireduce\_interval] The interval at which the statistics are
  collected by the root node.  This interval must be less than or
  equal to the output\_interval.  The way in which the statistics from
  different processors are combined is equivalent to \emph{appending}
  the time steps from each processor to form one long run. 
  Notes: the longer this is, the longer it will take for the                                                                                                          
     calculation to converge. The short this is, the more time                                                                                                           
     the calculation will spend shuttling data back and forth.                                                                                      
     This balance might be molecule specific since the communication                                                                                                     
     time depends on how much data to send.  

\item [mpipoll\_interval] The interval at which the worker nodes
  check to see if there are any commands from the root node.  The
  default value is 1. Notes: if this is too large, then the root processor might                                                                                                          
     end of spending a long time waiting. There isn't                                                                                                                    
     a significant penalty to making this small. Perhaps this                                                                                                            
     is the sort of parameter we want to remain opaque to the user.

\item [checkpoint] 1 if checkpoint files for restarting the calculation
  are to be written, 0 otherwise.  A checkpoint file contains the
  entire state of the calculation on one processor, and contains the
  rank of the processor in the file name.  Walker configurations,
  weights, and accumulated statistics are all written to the file.  

\item [checkpoint\_interval] The interval at which checkpoint files are
  to be written.  New checkpoint files overwrite old ones.  

\item [use\_available\_checkpoints] 1 if available checkpoint files are
  to be used in restarting the calculation.

\item [zero\_out\_checkpoint\_statistics] 1 if the the accumulated
  statistics in the checkpoint file are to be discarded, 0 if they are
  to be used.

\item [print\_transient\_properties] 1 if the accumulated statistics
  should be printed out during the calculation and 0 otherwise.

\item [print\_transient\_properties\_interval] The interval at which the
  accumulated statistics are printed out.

\item [write\_all\_energies\_out] 1 if the local energy for each
  walker at each time step should be written out and 0 otherwise.

\item [calculate\_bf\_density] 1 if the basis function density should be
  collected and written to a .density file.  This amounts to
  calculating the expectation value for each basis function, which can
  be used for fitting analytical densities to QMC
  simulations~\cite{AnalyticalDensity}.

\item [print\_configs] 1 if the electronic configurations are to be
  written to a .cfgs file in the temp directory, 0 if not.
  Configurations are always written if the wavefunction is being
  optimized. Notes: turning this on is required for wavefunction optimization.                                                                                                   
     However, it can *easily* take up all of your compute time since it                                                                                                  
     obviously requires extensive harddrive access. If you select optimization,                                                                                                    
     this will automatically be set to 1.                                                                                                    
     Also, depending on print\_config\_frequency and molecule size, these                                                                                                  
     files can easily get up to be several gigabytes in size. The files                                                                                                  
     can be output in ASCII or binary -- that option is set in code in QMCConfigIO.                                
     Eventually, we'll probably add HDF5 functionality, but last I (AGA) checked                                                                                         
     they were still working on their C++ API.  

\item [print\_config\_frequency] The interval at which configurations are
  written to the .cfgs file in the temp directory.  This number should
  be chosen carefully- if it is too small, the .cfgs file will become
  very large.  If it is too large, not enough configurations will be
  written to meaningfully compare sets of parameters.  

\end{description}

\subsection{Optimization}

\begin{description}

\item [optimize\_Psi] 1 if the parameters in the Jastrow function are to
  be optimized, 0 if not.  Optimization of the SCF part of the
  wavefunction (CI coefficients, orbitals, basis functions) is not
  implemented at this time. Notes: This will rerun the entire VMC calculation each time it                                                                                                      
     optimizes. That is, your calculation time might go up by a factor                                                                                                   
     of max\_optimize\_Psi\_steps, depending on convergence.

\item [max\_optimize\_Psi\_steps] One step is defined as follows- A series
  of configurations is generated with respect to the most accurate
  trial function available and correlated sampling configurations are
  written.  Then, using the configurations, several sets of parameters
  are evaluated and a new trial wavefunction is generated.  Default
  value is 10. Notes: Wavefunction convergence can take a long time, depending                                                                                                     
     on your Jastrow parameters.

\item [optimize\_Psi\_criteria] Objective function to use in
  comparing sets of parameters when optimizing a VMC wavefunction.
  Possibilities are \verb-energy_variance-, \verb-energy_average-,
  \verb-umrigar88-, which is described in~\cite{UmrigarWilsonWilkins1988}, and
  \verb-monkey_spank-, which is a combination of the variance of the
  local energy and a penalty function when the parameters become
  majorly different from the ones used in generating the correlated
  sampling configurations. Notes: for QMC, if your wavefunction is an eigenfunction, your                                                                                                      
     variance will be exactly zero. That's why you might want to minimize                                                                                                
     your variance. However, this isn't necessarily the best proceedure...  

\item [numerical\_derivative\_surface] Objective function to use
  when calculating derivatives with respect to the parameters for
  optimizing a VMC wavefunction.  Possibilities are the same as those
  for \emph{optimize\_Psi\_criteria}.

\item [optimize\_Psi\_barrier\_parameter] Objective function
  parameter used in forming a barrier around a valid region of
  parameter space for the \verb-monkey_spank- objective function.  A
  valid region of parameter space is one where the wavefunction is not
  majorly different from the one used to generate the walkers when
  correlated sampling is used. 

\item [optimize\_Psi\_method] Numerical optimization algorithm to
  use when optimizing a VMC wavefunction.  Possibilities are
  \verb-steepest_descent-, \verb-BFGSQuasiNewton-, and
  \verb-CKGeneticAlgorithm1-. 

\item [optimization\_max\_iterations] Maximum number of iterations
  to use when optimizing a VMC wavefunction with one set of correlated
  sampling configurations.  Default value is 100.

\item [optimization\_error\_tolerance] Tolerance used to determine
  when a numerical optimization with one set of correlated sampling
  configurations has converged.  Default value is 0.001.

\item [ch\_genetic\_algorithm\_1\_population\_size] Population size
  used when the CKGeneticAlgorithm1 is used to optimize a VMC
  wavefunction.  Default value is 30.

\item [ck\_genetic\_algorithm\_1\_initial\_distribution\_deviation]
  Standard deviation used when initializing the CKGeneticAlgorithm1.
  Default value is 1.

\item [ck\_genetic\_algorithm\_1\_mutation\_rate] Mutation rate used
  when the CKGeneticAlgorithm1 is used to optimize a VMC
  wavefunction.  Default value is 0.2.  
 
\item [line\_search\_step\_length] Algorithm used to determine the
  step length used when a line search algorithm is used to optimize a
  VMC wavefunction.  Possibilities are \verb-MikesBracketing- and
  \verb-Wolfe-.

\item [singularity\_penalty\_function\_parameter] Parameter used
  for the logarithmic barrier when D.R. Kent IV's algorithm for
  optimizing possibly singular Jastrow functions is used.  This
  parameter should be a small positive number.  The default value is
  1.0e-6.

\end{description}

\subsection{The Jastrow Function}

\begin{description}

\item [link\_Jastrow\_parameters] 1 if the parameters for up and down
  electrons are to be set equal to each other, 0 if not.  For example,
  if this flag is on, parameters for the the electron-up-electron-up
  and electron-down-electron-down correlation functions will be set
  equal to each other.  This reduces the number of parameters that
  have to be optimized and reduces spin contamination for singlet
  states.  

\item [scale\_Jastrow\_distances] 1 if distances in the Jastrow function
  are to be scaled by by $R=\frac{1-exp[-kr]}{k}$, where $k$ is a
  positive adjustable parameter.  As $r \rightarrow \infty$, 
  $R \rightarrow \frac{1}{k}$, which makes the function converge more
  rapidly with powers of $r$. 

\item [calc\_three\_body\_Jastrow] 1 if three body correlations are to be
  calculated.  The three body Jastrow is of the form used by
  Umrigar~\cite{ThreeBodyJastrow}. Not finished yet!

\end{description}

\subsection{Misc}

\begin{description}

\item [chip\_and\_mike\_are\_cool] C'mon, throw them a bone.

\end{description}

\section{The Jastrow Function}

Correlation functions are defined for each distinct pair of particles in the
molecule in the \&Jastrow section of the .ckmf file.  An example of a
correlation function is

\begin{verbatim}
ParticleTypes: Electron_Up Electron_Down
CorrelationFunctionType: FixedCuspPade
NumberOfParameterTypes: 2
NumberOfParametersOfEachType: 0 1
Parameters: 3.0
NumberOfConstantTypes: 1
NumberOfConstantsOfEachType: 1
Constants: 0.5
\end{verbatim}

This is a correlation function for an up electron and a down
electron.  A fixed cusp Pade function is used, which means that as
these two particles approach each other, the function will approach
the exact quantum mechanical solution for that collision.  The cusp
condition for opposite spin electrons is 0.5, while the cusp condition
for same spin electrons is 0.25 and the cusp condition for an electron
and a nucleus is the opposite of the charge of the nucleus.

The other possibilities function types are \verb-None-, in which the
particles are not correlated and \verb-Pade-, in which the cusp condition
is not fixed.  

The fixed cusp Pade correlation function is a fraction of polynomials,
and the parameters and constants are entered as follows:

The number of parameter types equals 2 because there are parameters in
the numerator and the denominator.  There are 0 numerator parameters
and 1 denominator parameter.  The denominator parameter is the first
degree coefficient of the polynomial, because the zeroth order
coefficient is always one.  The number of constant types is one
because there is only one constant, the cusp condition, which is the
first order term in the numerator.  This function, therefore, is 
\(
U_{\uparrow \downarrow}(r)=\frac{0.5r}{1+3.0r}
\)
.  The only adjustable parameter is the 3.0 in the denominator. 

More powers of $r$ can be added to the function by changing the
parameters part of the definition.  For example, the following
definition
\begin{verbatim}
ParticleTypes: Electron_Up Electron_Down
CorrelationFunctionType: FixedCuspPade
NumberOfParameterTypes: 2
NumberOfParametersOfEachType: 2 3
Parameters: 3.0 1.5 0.75 0.4 0.2
NumberOfConstantTypes: 1
NumberOfConstantsOfEachType: 1
Constants: 0.5
\end{verbatim}
would define the function 
\(
U_{\uparrow \downarrow}(r)
=\frac{0.5r+3.0r^{2}+1.5r^{3}}{1+0.75r+0.4r^{2}+0.2r^{3}}
\)
.  

For the Pade correlation function, the part of the definition dealing
with the constants is ignored, and the function is defined entirely
with the parameters.

When the wavefunction is optimized, parameters are varied and the
contants are not.  

%\section{Optimization Methods}

%\renewcommand{\baselinestretch}{1.0} \small
%\bibliographystyle{plain}
%\bibliography{beaver}

\end{document}

